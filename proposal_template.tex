%
\title{Gauntlet Run}
\author{
        Pragadeeshwar Dakshinamurthi
            \and
       Joshua Bornstein
           \and
       Patrick Negus
}
\date{\today}

\documentclass[11pt]{article}

\begin{document}

\maketitle

\section{Game description}


This is a puzzle/memory game, where the player controls an avatar which they need to guide along a path to the end zone. The path is visible for a certain amount of time before it disappears, after which the player needs to use their memory to finish the level. The player is given a few mechanics (such as a boost) at their disposal which helps them traverse quickly but at the same time requiring more precise inputs.\\

The educational component here is the memory aspect; we also intend for this game to require precise timing and inputs. The target audience is to be players of all ages. Finally, we intend to use Unity for this project.

\subsection{Details}
\begin{itemize}
\item What are you trying to teach through this game?

We want to teach visual memory skills here, as well as challenging the player to innovate when it comes to traversing the levels. We don't intend for every level in our game to be beaten via the "direct path" to the exist, though the player is absolutely welcome to do so.

\item 3D world and characters (e.g., will it be an open/closed environment? what type of objects/entities will it contain? Will it involve one or multiple human players (i.e., networking)? Will there be any AI characters?)
The game will be a series of levels (potentially auto-generated) each consisting of a 3D world containing a 2d checkerboarded or otherwise patterned terrain. The user will have the ability to control player character and an overhead camera. There will be one user player character, and 0 or more AI turrets. 

\item Game mechanics (e.g., what the player(s) will do? How does the environment change or what other entities do in response to the player’'s actions?)
At the begining of the level, a path of connected squares leading from the start to the finish square will light up. Then those lights will disappear and the user will attempt to walk they're player charcter along the pass to the finish. If they step off the indicated path, the square or area of terrain will briefly light up red and they will be transported back to the start square, where the path indicator lights will flash again. When the player reaches the end goal it will briefly light up green and the player will be transported to the start of the next level.  

\item Animation (e.g., will you use physics-based animation, rigid body simulation, hard-coded animations, keyframe animations, etc?)


\item User interface and sound (will the game have any menus? what input devices/sensors do you plan to use? Will it work on a mobile phone? Will there be any sound effects?)
\end{itemize}

If you want to show any images (e.g., sketches, screenshots, etc) to explain your plan better, you are more than welcome to do so! Anything that will help me understand what you are planning to do will be useful.  


\paragraph{Split of the work.}
Provide  a rough breakdown of the tasks each group member will focus on. You do not need to include this paragraph if you do not plan to work in groups. 
Our team has 3 members, Pragadeeshwar, Patrick, and Joshua.
Pragadeeshwar is going to focus on the design of the world and the characters, items, animations in it. 
Josh is going to focus on writing the scripts that control the gameflow, and making auto-level-generation
Patrick is going to focus on the UI aspects of playing the game as well as help making animations, and configiring scene views.  
All group members contribute ideas about level design and general gameplay. 


\end{document}
